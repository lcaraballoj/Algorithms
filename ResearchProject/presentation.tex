\documentclass{beamer}
\usetheme{Madrid}
\usecolortheme{beaver}
\usepackage[utf8]{inputenc}
\usepackage{amsmath}
\usepackage{amssymb}
\usepackage{amsthm}
\usepackage{comment}
\usepackage{multicol}
\usepackage{csquotes}
\usepackage{booktabs}
\usepackage{color, colortbl}
\usepackage{polynom}
\usepackage{tcolorbox}

\definecolor{darkRed}{cmyk}{0,1,.87,.22}
\definecolor{lightRed}{cmyk}{0,.67,.52,0}

\newcolumntype{R}{>{\columncolor{darkRed}}c}

\title{0-1 Knapsack Problem}
\author[Linnea Caraballo]{{\large Linnea Caraballo }}
\institute[Sacred Heart University]{CS 341 Professor Pinto}
\date{May 7, 2022}

\begin{document}
	
	\maketitle
	
	\begin{frame}{What is the 0-1 Knapsack Problem?}
		\begin{itemize}
			\item A problem that requires one to fill a knapsack that has a specific weight capacity with items that have two conditions, value and weight.
			\item \textbf{Goal:} Fill the knapsack in a way that optimizes the value fo the bag, but does no exceed the weight capacity. 
			\item \textbf{Use Case:} Resource allocation problems. 
		\end{itemize}
	\end{frame}

	\begin{frame}{Brute Force}
		\begin{itemize}
			\item Let $S = \{item_1, item_2, item_3, ..., item_k\}, w_i = \mbox{weight of } item_i, v_i = \mbox{value of } item_i, \mbox{ and } W = \mbox{weight capacity of knapsack}$. 
			\item Brute force requires us to consider all subsets of $S$ which we will call $N$.
			\item We first need to calculate every single $N$
			\item We will then discarde those that exceed $W$, the weight limit of the knapsack. 
			\item This leaves us to then find the subset with the largest value. 
			\begin{tcolorbox}
				We will need to calculate $2^n$ subsets, so $O(2^n)$ which is undesirable.
			\end{tcolorbox}
		\end{itemize}
	\end{frame}

	\begin{frame}{Greedy Approach 1}
		\begin{itemize}
			\item Pick the highest value first and then continue in decreasing order of item values.
				\begin{table}[h!]
					\begin{center}
						\caption{Example from CodesDope \cite{noauthor_knapsack_nodate-2}}
						\label{tab:example1}
						\begin{tabular}{c|c|c}
							\toprule % <-- Toprule here
							\textbf{Item Number} & \textbf{Weight} & \textbf{Value}\\
							$item_i$ & $w_i$ & $p_i$ \\
							\midrule % <-- Midrule here
							1 & 12 & 100 \\
							2 & 32 & 200 \\
							3 & 30 & 50 \\
							4 & 5 & 60 \\
							5 & 34 & 150
						\end{tabular}
					\end{center} \vspace{12pt}
				$$
				W = 50
				$$
			\end{table}
			\item Using this method gets us $N_1 = \{item_2, item_1\}$ in that order for a total value of $300$ and weight of $44$ [\textbf{Highest value: 360}].
		\end{itemize}
	\end{frame}
	
	\begin{frame}{Greedy Approach 2}
		\begin{itemize}
			\item Pick the lowest weight item first and then go in ascending order of weight.
				\begin{table}[h!]
				\begin{center}
					\caption{Example from CodesDope \cite{noauthor_knapsack_nodate-2}}
					\label{tab:example1.5}
					\begin{tabular}{c|c|c}
						\toprule % <-- Toprule here
						\textbf{Item Number} & \textbf{Weight} & \textbf{Value}\\
						$item_i$ & $w_i$ & $p_i$ \\
						\midrule % <-- Midrule here
						1 & 12 & 100 \\
						2 & 32 & 200 \\
						3 & 30 & 50 \\
						4 & 5 & 60 \\
						5 & 34 & 150
					\end{tabular}
				\end{center} \vspace{12pt}
				$$
				W = 50
				$$
				\end{table}
			\item Using this method gets us $N_2 = \{item_4, item_3, item_1\}$, which gets us a total value of $210$ and a total weight of $47$ [\textbf{Highest value: 360}].
		\end{itemize}	
	\end{frame}
	
	\begin{frame}{Greedy Approach 3}
		\begin{itemize}
			\item First pick the items with the largest value per weight.
			\begin{table}[h!]
				\begin{center}
					\caption{Example from CodesDope \cite{noauthor_knapsack_nodate-2}}
					\label{tab:example1.6}
					\begin{tabular}{c|c|c}
						\toprule % <-- Toprule here
						\textbf{Item Number} & \textbf{Weight} & \textbf{Value}\\
						$item_i$ & $w_i$ & $p_i$ \\
						\midrule % <-- Midrule here
						1 & 5 & 50 \\
						2 & 10 & 60 \\
						3 & 20 & 140 
					\end{tabular}
				\end{center} \vspace{12pt}
				$$
				W = 30
				$$
			\end{table}
			\item Using the same method we get a value of $190$ [\textbf{Highest Value: $200$}]. 
		\end{itemize}
	\end{frame}

	\begin{frame}{Dynamic Programming: Bottom-Up}
		\begin{itemize}
			\item \textbf{Goal:} Put values into a table where the rows represent the weight limit and the columns represent the items.
			\item This means that if we pick a cell $(i, j)$ it would contain the optimal value for the first $i$ items for a weight limit of $j$.
			\item \textbf{Order:} $O(n*W)$
		\end{itemize}
	\end{frame}

	\begin{frame}{Bottom-Up Example Setup}
		\begin{table}[h!]
			\begin{center}
				\caption{Example from CodesDope \cite{noauthor_knapsack_nodate-2}}
				\label{tab:example2}
				\begin{tabular}{c|c|c}
					\toprule % <-- Toprule here
					\textbf{Item Number} & \textbf{Weight} & \textbf{Value}\\
					$item_i$ & $w_i$ & $p_i$ \\
					\midrule % <-- Midrule here
					1 & 3 & 8 \\
					2 & 2 & 3 \\
					3 & 4 & 9 \\
					4 & 1 & 6 
				\end{tabular}
			\end{center} \vspace{12pt}
			$$
			W = 5
			$$
		\end{table}
	\end{frame}

	\begin{frame}{Bottom-Up Example Cont.}
		\begin{table}[h!]
			\begin{center}
				\caption{Example from CodesDope \cite{noauthor_knapsack_nodate-2}}
				\label{tab:example2-step1}
				\begin{tabular}{R|c|c|c|c|c|c}
					\toprule % <-- Toprule here
					\rowcolor{darkRed}
					\cellcolor{lightRed}\textcolor{white}{\textbf{$w \rightarrow$}} & \textcolor{white}{\textbf{0}} & \textcolor{white}{\textbf{1}} & \textcolor{white}{\textbf{2}} & \textcolor{white}{\textbf{3}} & \textcolor{white}{\textbf{4}} & \textcolor{white}{\textbf{5}}\\
					\rowcolor{darkRed}
					\cellcolor{lightRed}\textcolor{white}{$item_i \downarrow$} & & & & & & \\
					\midrule % <-- Midrule here
					\textcolor{white}{0} & 0 & 0 & 0 & 0 & 0 &  0 \\
					\hline
					\textcolor{white}{1} & 0 & & & & & \\
					\hline
					\textcolor{white}{2} & 0 & & & & & \\
					\hline
					\textcolor{white}{3} & 0 & & & & & \\
					\hline
					\textcolor{white}{4} & 0 & & & & & 
				\end{tabular}
			\end{center} \vspace{12pt}
			Observe that for a maximum weight of $0$ that we cannot take any items, so the whole column $j = 0$ would have a total value of $0$. Similarly note how if we take 0 items that the total value would be $0$, so the whole row $i = 0$ is also $0$.
		\end{table}
	\end{frame}

	\begin{frame}{Bottom-Up Example Cont.}
		\begin{table}[h!]
			\begin{center}
				\caption{Example from CodesDope \cite{noauthor_knapsack_nodate-2}}
				\label{tab:example2-step2}
				\begin{tabular}{R|c|c|c|c|c|c}
					\toprule % <-- Toprule here
					\rowcolor{darkRed}
					\cellcolor{lightRed}\textcolor{white}{\textbf{$w \rightarrow$}} & \textcolor{white}{\textbf{0}} & \textcolor{white}{\textbf{1}} & \textcolor{white}{\textbf{2}} & \textcolor{white}{\textbf{3}} & \textcolor{white}{\textbf{4}} & \textcolor{white}{\textbf{5}}\\
					\rowcolor{darkRed}
					\cellcolor{lightRed}\textcolor{white}{$item_i \downarrow$} & & & & & & \\
					\midrule % <-- Midrule here
					\textcolor{white}{0} & 0 & 0 & 0 & 0 & 0 &  0 \\
					\hline
					\textcolor{white}{1} &  0 & \cellcolor{lightRed}\textcolor{white}{0} & \cellcolor{lightRed}\textcolor{white}{0} & \cellcolor{lightRed}\textcolor{white}{8} & \cellcolor{lightRed}\textcolor{white}{8} & \cellcolor{lightRed}\textcolor{white}{8} \\
					\hline
					\textcolor{white}{2} & 0 & & & & \\
					\hline
					\textcolor{white}{3} & 0 & & & & & \\
					\hline
					\textcolor{white}{4} & 0 & & & & &
				\end{tabular}
			\end{center} \vspace{12pt}
		\end{table}
		\textbf{Recall:} $w_1 = 3$ with a value $v_1 = 8$. This means that for all $w < 3$ that we cannot take an item, so these cells have a $0$. For any $w \geq 3$ we can put item 1 into our knapsack, so these cell values will be $8$
	\end{frame}

	\begin{frame}{Bottom-Up Example Cont.}
		\begin{table}[h!]
			\begin{center}
				\caption{Example from CodesDope \cite{noauthor_knapsack_nodate-2}}
				\label{tab:example2-fullTable}
				\begin{tabular}{R|c|c|c|c|c|c}
					\toprule % <-- Toprule here
					\rowcolor{darkRed}
					\cellcolor{lightRed}\textcolor{white}{\textbf{$W \rightarrow$}} & \textcolor{white}{\textbf{0}} & \textcolor{white}{\textbf{1}} & \textcolor{white}{\textbf{2}} & \textcolor{white}{\textbf{3}} & \textcolor{white}{\textbf{4}} & \textcolor{white}{\textbf{5}}\\
					\rowcolor{darkRed}
					\cellcolor{lightRed}\textcolor{white}{$item_i \downarrow$} & & & & & & \\
					\midrule % <-- Midrule here
					\textcolor{white}{0} & 0 & 0 & 0 & 0 & 0 &  0 \\
					\hline
					\textcolor{white}{1} & 0 & 0 & 0 & 8 & 8 & 8 \\
					\hline
					\textcolor{white}{2} & 0 & 0 & 3 & 8 & 8 & 11 \\
					\hline
					\textcolor{white}{3} & 0 & 0 & 3 & 8 & 8 & 11 \\
					\hline
					\textcolor{white}{4} & 0 & 6 & 6 & 9 & 14 & 15
				\end{tabular}
			\end{center} \vspace{12pt}
		\end{table}
		\textbf{Observe:} $F(4,1) = 6$ because $w_4 = 1$ and $v_4 = 6$, so we can carry that item with a $w = 1$. Also observe that $F(4,5) = 15$ becuase we can take $item_3 \mbox{ and } item_4 \mbox{, such that } w_3 + w_4 = 5 \mbox{ and } v_3 + v_4 = 9 + 6$.
	\end{frame}

	\begin{frame}{Refinement of Dynamic Programming Approach (Recursion)}
		\begin{itemize}
			\item The order using bottom-up can exceed $O(2^n)$ is for example given $W = 2! \mbox{ leading to } O(n*n!)$.
			\item We can instead simply consider $F[i][w] \mbox{ and } F[i-1][W] \mbox{ and } F[i-1][W - w_i] \mbox{ where }n$ is the number of items and $i$ is the item we are talking about/the row we are in.
			\item We start from $n$ and end when either $n = 1$ or $w \leq 0$. Leaving us with the equation. 
			
			\small\begin{equation*}
				F[i][W] = \left\{
				\begin{array}{ll}
					maximum(P[i-1][W], \ v_i + P[i-1][w-w_i]) & \mbox{if }  w_i \leq W \\
					P[n-1][W]& \mbox{if } w_i > W
				\end{array}
				\right.
			\end{equation*}
			\item \textbf{Order:} $O(minimize((n*W), (2^n)))$
		\end{itemize}
	\end{frame}
	
	\begin{frame}{Refinement of Dynamic Programming Example Setup}
		\begin{table}[h!]
			\begin{center}
				\caption{Example from \textit{Foundations of Algorithms} \cite{neopolitan_foundations_nodate}}
				\label{tab:example3}
				\begin{tabular}{c|c|c}
					\toprule % <-- Toprule here
					\textbf{Item Number} & \textbf{Weight} & \textbf{Value}\\
					$item_i$ & $w_i$ & $p_i$ \\
					\midrule % <-- Midrule here
					1 & 5 & 50 \\
					2 & 10 & 60 \\
					3 & 20 & 140 \\ 
				\end{tabular}
			\end{center} \vspace{12pt}
			$$
			W = 30
			$$
		\end{table}
	\end{frame}

	\begin{frame}{Refinement of Dynamic Programming Example Cont.}
		\begin{itemize}
			\item \textbf{Determine the entries need:} 
				\begin{itemize}
					\item Row 3 we need $F[3][30]$
					\item Row 2 we compute $F[3][30] \mbox{, thus we need } F[2][30] \mbox{ and } F[3-1][30-w_3] = F[2][10]$
					\item Row 2 we compute $F[2][30]$, so we need $F[1][30] \mbox{ and } F[1][20]$ 
					\item Row 1 we compute $F[1, 10] \mbox{ and } F[1][0]$
					\item Stop because $n = 1$ as well as $w \leq 0$
				\end{itemize}
		\end{itemize}
	\end{frame}

	\begin{frame}{Refinement of Dynamic Programming Example Cont.}
		\textbf{Step 1: }
		\begin{align*}
			F[1][w] &= 
			\begin{cases}
				maximum(P[0][w], \ 50 + P[0][w-5]) & \mbox{if }  w_1 = 5 \leq w \\
				P[0][5] & \mbox{if } w_1 = 5 > w
			\end{cases} \\
			&= 
			\begin{cases}
				50 & \mbox{if }  5 \leq w \\
				0 & \mbox{if } 5 > w
			\end{cases}
		\end{align*}
		
		\textbf{Step 2: }
		\begin{align*}
			F[2][10] &= 
			\begin{cases}
				maximum(P[1][10], \ 60 + P[1][0]) & \mbox{if }  w_2 = 10 \leq 10 \\
				P[1][10] & \mbox{if } w_2 = 10 > 10
			\end{cases} \\
			&= 60
		\end{align*}
	\end{frame}

	\begin{frame}{Refinement of Dynamic Programming Cont. (Steps)}
				\textbf{Step 3: }
		\begin{align*}
			F[2][30] &= 
			\begin{cases}
				maximum(P[1][30], \ 60 + P[1][20]) & \mbox{if }  w_2 = 10 \leq 30 \\
				P[1][30] & \mbox{if } w_2 = 10 > 30
			\end{cases} \\
			&= 60 + 50 = 110
		\end{align*}
		
		\textbf{Step 4: }
		\begin{align*}
			F[3][30] &= 
			\begin{cases}
				maximum(P[2][30], \ 140 + P[2][10]) & \mbox{if }  w_3 = 20 \leq 30 \\
				P[2][30] & \mbox{if } w_3 = 20 > 30
			\end{cases} \\
			&= 140 + 60 = 200
		\end{align*}
	\end{frame}
	
	\begin{frame}{Resources Used}
		\bibliographystyle{plain}
		\bibliography{Algorithms} 
		\textcolor{white}{\cite{martin_01_2020}\cite{neopolitan_foundations_nodate}\cite{noauthor_knapsack_nodate}\cite{noauthor_knapsack_nodate-2}}
	\end{frame}
\end{document}
